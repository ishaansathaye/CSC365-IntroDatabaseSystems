\documentclass[twoside]{article}
\setlength{\oddsidemargin}{0.25 in}
\setlength{\evensidemargin}{-0.25 in}
\setlength{\topmargin}{-0.6 in}
\setlength{\textwidth}{6.5 in}
\setlength{\textheight}{8.5 in}
\setlength{\headsep}{0.75 in}
\setlength{\parindent}{0 in}
\setlength{\parskip}{0.1 in}

%
% add packages if you need them here:
%

\usepackage{amsmath,amsfonts,graphicx}
\usepackage{algorithm,caption}
\usepackage[noend]{algpseudocode}
\usepackage{url}
%
% the following macro is used to generate the header.
%
\newcommand{\notes}[4]{
   \pagestyle{myheadings}
   \thispagestyle{plain}
   \newpage
   \setcounter{page}{1}
   \noindent
   \begin{center}
   \framebox{
      \vbox{\vspace{2mm}
    \hbox to 6.28in { {\bf CSC 365-07: Introduction to Database Systems
		\hfill Spring 2023} }
       \vspace{4mm}
       \hbox to 6.28in { {\Large \hfill Lecture 1:  Database and DBMS \hfill} }
       \vspace{2mm}
       \hbox to 6.28in { {\it Instructor: Alex Dekhtyar \hfill Ishaan Sathaye} }
      \vspace{2mm}}
   }
   \end{center}
}

% use these for theorems, lemmas, proofs, etc.
\newtheorem{theorem}{Theorem}
\newtheorem{lemma}[theorem]{Lemma}
\newtheorem{proposition}[theorem]{Proposition}
\newtheorem{claim}[theorem]{Claim}
\newtheorem{corollary}[theorem]{Corollary}
\newtheorem{definition}[theorem]{Definition}
\newenvironment{proof}{{\bf Proof:}}{\hfill\rule{2mm}{2mm}}

\begin{document}
\notes




\section*{Introduction}
Definition of a database and DBMS in Professor Notes.

\newpage
\hfill \break
\framebox{
    \vbox{\vspace{2mm}
    \hbox to 6.28in { {\bf CSC 365-07: Introduction to Database Systems
    \hfill Spring 2023} }
    \vspace{4mm}
    \hbox to 6.28in { {\Large \hfill Lecture 2:  Relational Data Model \hfill} }
    \vspace{2mm}
    \hbox to 6.28in { {\it Instructor: Alex Dekhtyar \hfill Ishaan Sathaye} }
    \vspace{2mm}}
}

\section*{Relational Data Model}
\begin{definition}
    Relational data model is an approach to organizing collections of data
\end{definition}

\begin{itemize}
    \item Relation
    \begin{itemize}
        \item Relational Table $\longrightarrow$ \textbf{Name + Schema}
        \begin{itemize}
            \item Schema: List of attribute name + attribute type pairs
        \end{itemize}
    \end{itemize}
    \item Relational Database $\longrightarrow$ \textbf{Collection of Relations tables}
    \item \textbf{Table Instance}: set of records with instantiated values of the attributes
    \begin{itemize}
        \item Finite
        \item Records, rows, tuples
    \end{itemize}
\end{itemize}

One unit of data is called a \textbf{datum}.

Object, entity, event: description of one object, entity, event
\begin{itemize}
    \item \textbf{Records} consist of attributes or fields (rows in the 
    table).
    \item \textbf{Attributes} is a named container for a value of a specific type.
\end{itemize}

\subsection*{Database Table Constraint}
\begin{definition}
    Limitations of table instances
\end{definition}
\begin{itemize}
    \item \textbf{Candidate Key}: set or lists of attributes that uniquely
    define a record in a table, \textbf{minimal such set of attributes},
    made up of multiple attributes sometimes.
    \begin{itemize}
        \item \textbf{Every attribute is necessary.}
    \end{itemize}
\end{itemize}

\section*{Examples}
\subsection*{CSC 365 Example}

Course Object:
\begin{itemize}
    \item Prefix: CSC $\longrightarrow$ \textbf{String}
    \item Course \#: 365 $\longrightarrow$ \textbf{Integer}
    \item Name: Introduction to Database Systems $\longrightarrow$ \textbf{String}
    \item Description: Basic Principles, ... $\longrightarrow$ \textbf{String}
    \item Units: 4 $\longrightarrow$ \textbf{Integer}
\end{itemize}
Department Object:
\begin{itemize}
    \item Name: Computer Science and Software Engineering
    \item Abbreviation: CSSE
    \item Building: 14
    \item Room: 245
    \item College: CENG
\end{itemize}

Stringing these objects together based on relationship would 
make a \textbf{network model}.

\subsection*{Schema Example}
\begin{verbatim}
    Course(Prefix String, Course# Integer, Name String, Description
    String, Units Integer)
\end{verbatim}

\begin{table}[h]
    \centering
    \begin{tabular}{|c|c|c|c|c|}
        \hline
        Prefix & Course\# & Name & Description & Units \\
        \hline
        CSC & 365 & Introduction to Database Systems & Basic Principles, ... & 4 \\
        \hline
        CSC & 357 & Systems Programming & ... & 4 \\
        \hline
    \end{tabular}
\end{table}

\begin{verbatim}
    Department(Name, College, Building, Room): Department would also have a table as well.
\end{verbatim}

\newpage
\hfill \break 
\framebox{
    \vbox{\vspace{2mm}
    \hbox to 6.28in { {\bf CSC 365-07: Introduction to Database Systems
    \hfill Spring 2023} }
    \vspace{4mm}
    \hbox to 6.28in { {\Large \hfill Lecture 3:  RDM Cont. \hfill} }
    \vspace{2mm}
    \hbox to 6.28in { {\it Instructor: Alex Dekhtyar \hfill Ishaan Sathaye} }
    \vspace{2mm}}
}

\section*{Relational Data Model}

What makes a record unique?
\begin{itemize}
    \item \textbf{Superkey}: any set of attributes that uniquely defines a record in a table
    \item \textbf{Primary Key}: candidate key chosen by you
\end{itemize}

\newpage
\hfill \break 
\framebox{
    \vbox{\vspace{2mm}
    \hbox to 6.28in { {\bf CSC 365-07: Introduction to Database Systems
    \hfill Spring 2023} }
    \vspace{4mm}
    \hbox to 6.28in { {\Large \hfill Lecture 4: SQL DDL and DML \hfill} }
    \vspace{2mm}
    \hbox to 6.28in { {\it Instructor: Alex Dekhtyar \hfill Ishaan Sathaye} }
    }
}

\section*{MySQL Access}
\begin{enumerate}
    \item Server Address = host: \textbf{mysql.labthreesixfive.com}
    \item Port: 3306
    \item username
    \item password
\end{enumerate}

MySQL Database
\begin{itemize}
    \item Namespace
    \item Collection of Tables
    \item Set of Permissions
\end{itemize}

\section*{Case Sensitivity}
\subsection*{Case Sensitive}
\begin{itemize}
    \item Table Names
    \item Database Names
\end{itemize}
\subsection*{Not Case Sensitive}
\begin{itemize}
    \item Attribute Names
    \item SQL Keywords
\end{itemize}

\section*{Types}
\begin{itemize}
    \item \textbf{Numeric Types}
    \begin{itemize}
        \item \textbf{Integer Types}
        \begin{itemize}
            \item TINYINT
            \item SMALLINT
            \item MEDIUMINT
            \item \textbf{INT}
            \item BIGINT
        \end{itemize}
        \item \textbf{Floating Point Types}
        \begin{itemize}
            \item \textbf{FLOAT}
            \item \textbf{DOUBLE(P, D)}
            \item \textbf{DECIMAL}
        \end{itemize}
    \end{itemize}
    \item \textbf{String Types}
    \begin{itemize}
        \item \textbf{Character Types}
        \begin{itemize}
            \item \textbf{CHAR(N)} $\longrightarrow$ \textbf{Fixed Length}
            \item \textbf{VARCHAR(N)} $\longrightarrow$ \textbf{Variable Length}
            \item TINYTEXT
            \item \textbf{TEXT} $\longrightarrow$ for storing large amounts of text
            \item MEDIUMTEXT
            \item LONGTEXT
        \end{itemize}
    \end{itemize}
    \item \textbf{Date and Time Types}
    \begin{itemize}
        \item \textbf{Date Types}
        \begin{itemize}
            \item DATE
            \item DATETIME
            \item TIMESTAMP
            \item TIME
            \item YEAR
        \end{itemize}
    \end{itemize}
\end{itemize}

\section*{Data Definition Language (DDL)}

\textbf{Commands from DDL act upon the schema}
\begin{itemize}
    \item CREATE TABLE
    \item DROP TABLE
    \item ALTER TABLE
\end{itemize}

\subsection*{Define a Relational Table}
Aspects needed to define a table:
\begin{itemize}
    \item Table Name
    \item Attributes: Name + Type
    \item Constraints
\end{itemize}

\begin{verbatim}
    CREATE TABLE <table_name> (
        <attribute_name> <sql_type> [<single_line_constraints>],
        ...,
        <attribute_name> <sql_type> [<single_line_constraints>] [,
        <constraints>[,
        <constraints>]
    ]);
\end{verbatim}

\section*{Data Manipulation Language (DML)}

\textbf{Commands from DML act upon the instance.}
\begin{itemize}
    \item INSERT
    \item DELETE
    \item UPDATE
\end{itemize}

\subsection*{Inserting Data}
\begin{verbatim}
    INSERT INTO <table_name>(<attribute_name>, ...)
        VALUES (<value>, ...);
\end{verbatim}
Supply values in order of attribute declarations in CREATE TABLE statement.
Can omit the attribute names if values supplied are in the same order. If need to omit
a value then omit that attribute name as well.

\section*{More on Constraints}
\begin{itemize}
    \item \textbf{[NOT] NULL} - attribute cannot be null
    \item \textbf{UNIQUE}
    \item \textbf{PRIMARY KEY}
    \item \textbf{FOREIGN KEY}
    \item \textbf{DEFAULT $<$exp$>$} - default value for attribute
    \item \textbf{AUTO\textunderscore INCREMENT} - means that the attribute is an integer 
    and is automatically incremented
\end{itemize}

\section*{Lab 2}
MySQL Server
\begin{itemize}
    \item LabThreeSixFive.com
    \item mysql command line client
    \item IDE (DatGrip)
    \item mysql connectivity from Python
\end{itemize}

Lab 2 uses Create Table, Drop Table, and Insert.

\section*{Code from Lab}
\begin{verbatim}
show tables

CREATE TABLE Departments (
    
    DeptId INT PRIMARY KEY, 
    Abbr VARCHAR(20) UNIQUE, -- UNIQUE makes candidate key
    Name VARCHAR(128) UNIQUE,
    College CHAR(10),
    Building INT,
    Room CHAR(6),
    -- set multiple candidate keys at the bottom
    UNIQUE(Building, Room),
    -- foreign key always a separate line statement:
    -- FOREIGN KEY(College) REFERENCES colleges(abbr)
    
);

describe colleges;
SELECT * FROM colleges;

show CREATE TABLE colleges;

show CREATE TABLE Departments;

INSERT INTO Departments 
    VALUES(1, 'CSSE', 'Computer Science and Software Engineering', 'CENG', 14, '245');
    
INSERT INTO Departments(DeptId, Abbr, Name, College, Building, Room)
    VALUES(1, 'CSSE', 'Computer Science and Software Engineering', 'CENG', 14, '245');

\end{verbatim}

\newpage
\hfill \break 
\framebox{
    \vbox{\vspace{2mm}
    \hbox to 6.28in { {\bf CSC 365-07: Introduction to Database Systems
    \hfill Spring 2023} }
    \vspace{4mm}
    \hbox to 6.28in { {\Large \hfill Lecture 5: DDL and DML Continued \hfill} }
    \vspace{2mm}
    \hbox to 6.28in { {\it Instructor: Alex Dekhtyar \hfill Ishaan Sathaye} }
    }
}

\section*{DML}
\subsection*{Updating Data}
\begin{verbatim}
    UPDATE <table_name>
        SET <attribute_name> = <value>
        WHERE <condition>;
\end{verbatim}

\subsubsection*{Example}
\begin{verbatim}
    UPDATE colleges
        SET abbr = 'COSAM'
        WHERE abbr = 'COASM'
\end{verbatim}

WHERE clause is a filter that determines which rows are updated.

\subsection*{Deleting Data}
\begin{verbatim}
    DELETE FROM <table_name>
        WHERE <condition>;
\end{verbatim}

\section*{DDL}
\subsection*{Altering Tables}
\begin{verbatim}
    ALTER TABLE <table_name>
        <Command> <parameters>;
\end{verbatim}
\subsubsection*{Commands}
\begin{itemize}
    \item ADD - add a column/attribute/key
    \item DROP
    \item MODIFY
    \item RENAME
\end{itemize}
\subsubsection*{Parameters}
\begin{itemize}
    \item COLUMN
    \item CONSTRAINT
    \item FOREIGN KEY
    \item PRIMARY KEY
    \item UNIQUE
\end{itemize}

Adding an attribute, dropping/adding a constraint, renaming a table, disable/enabling
keys, and modifying attributes examples are in this professor notes: \url{4-SQLDDLDML.pdf}

\newpage
\hfill \break 
\framebox{
    \vbox{\vspace{2mm}
    \hbox to 6.28in { {\bf CSC 365-07: Introduction to Database Systems
    \hfill Spring 2023} }
    \vspace{4mm}
    \hbox to 6.28in { {\Large \hfill Lecture 5: DDL and DML Continued \hfill} }
    \vspace{2mm}
    \hbox to 6.28in { {\it Instructor: Alex Dekhtyar \hfill Ishaan Sathaye} }
    }
}

\end{document}