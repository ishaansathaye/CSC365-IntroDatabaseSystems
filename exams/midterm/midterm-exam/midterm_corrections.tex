\documentclass{article}

\usepackage{color,fancyhdr,ifthen,amssymb,amsfonts,amsmath}

\pagestyle{fancy}
\setlength{\topmargin}{-.5in}
\setlength{\textheight}{9in}
\setlength{\oddsidemargin}{0in}
\setlength{\evensidemargin}{0in}
\setlength{\textwidth}{6.5in}
\setlength{\headwidth}{\textwidth}
\parindent=0in

\newcommand{\headandfoot}[3]{\lhead{#1}\chead{#2}\rhead{\ifthenelse{\isodd{
    \thepage}}{Ishaan Sathaye
 {\hspace{.25in}}}{}}}

\headandfoot{Midterm Corrections}{ CSC 365}

\begin{document}

\subsection*{Problem 1 (b)}
Table R has a primary A and contains 17,000 tuples. Table S has a primary key
B and an attribute C, which is a foreign key referencing R. Table S contains 
3,457 tuples. How many tuples will $R\bowtie_{R.A=S.C}S$ contain? \newline

\textbf{Answer:} 3,457 tuples. The join will only contain tuples where the
value of the A column is equal to the value of the C column in S.

\subsection*{Problem 2 (a)}
Write SQL commands to create all three tables. (Make sure your SQL commands
include ALL necessary information.)

\begin{verbatim}
CREATE TABLE Musicians {
    Id INT PRIMARY KEY,
    Name CHAR(20),
    Country CHAR(20)
}
CREATE TABLE CD {
    Id INT PRIMARY KEY,
    Musician INT,
    Title CHAR(20),
    Year INT,
    Label CHAR(20),
    UNIQUE(Musician, Title),
    FOREIGN KEY(Musician) REFERENCES Musicians(Id)
}
CREATE TABLE Songs {
    CD INT,
    TrackNo INT,
    Title CHAR(20),
    DURATION TIME,
    PRIMARY KEY(CD, TrackNo),
    FOREIGN KEY(CD) REFERENCES CD(Id)
}
\end{verbatim}


\end{document}
