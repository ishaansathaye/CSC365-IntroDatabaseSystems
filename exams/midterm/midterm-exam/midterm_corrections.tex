\documentclass{article}

\usepackage{color,fancyhdr,ifthen,amssymb,amsfonts,amsmath}

\pagestyle{fancy}
\setlength{\topmargin}{-.5in}
\setlength{\textheight}{9in}
\setlength{\oddsidemargin}{0in}
\setlength{\evensidemargin}{0in}
\setlength{\textwidth}{6.5in}
\setlength{\headwidth}{\textwidth}
\parindent=0in

\newcommand{\headandfoot}[3]{\lhead{#1}\chead{#2}\rhead{\ifthenelse{\isodd{
    \thepage}}{Ishaan Sathaye
 {\hspace{.25in}}}{}}}

\headandfoot{Midterm Corrections}{ CSC 365}

\begin{document}

\subsection*{Problem 1 (b)}
Table R has a primary A and contains 17,000 tuples. Table S has a primary key
B and an attribute C, which is a foreign key referencing R. Table S contains 
3,457 tuples. How many tuples will $R\bowtie_{R.A=S.C}S$ contain? \newline

\textbf{Answer:} 3,457 tuples. The join will only contain tuples where the
value of the A column is equal to the value of the C column in S.

\subsection*{Problem 2}
\subsubsection*{(a)}
Write SQL commands to create all three tables. (Make sure your SQL commands
include ALL necessary information.)

\begin{verbatim}
CREATE TABLE Musicians {
    Id INT PRIMARY KEY,
    Name CHAR(20),
    Country CHAR(20)
}
CREATE TABLE CD {
    Id INT PRIMARY KEY,
    Musician INT,
    Title CHAR(20),
    Year INT,
    Label CHAR(20),
    UNIQUE(Musician, Title),
    FOREIGN KEY(Musician) REFERENCES Musicians(Id)
}
CREATE TABLE Songs {
    CD INT,
    TrackNo INT,
    Title CHAR(20),
    DURATION TIME,
    PRIMARY KEY(CD, TrackNo),
    FOREIGN KEY(CD) REFERENCES CD(Id)
}
\end{verbatim}

\subsubsection*{(b)}
Write SQL commands that add two attributes to the Songs table: for the author 
of the music, and one for the author of the lyrics. Write a SQL command to set
the author of both music and lyrics for the song Airport to be Hammill. \newline

\begin{verbatim}
    ALTER TABLE Songs
        ADD MusicAuthor CHAR(20),
        ADD LyricsAuthor CHAR(20);
    UPDATE Songs
        SET MusicAuthor = 'Hammill', LyricsAuthor = 'Hammill'
        WHERE Title = 'Airport';
\end{verbatim}

\subsubsection*{(c)}
Write a single SQL statement that keeps in the CDs table only the information
about (a) albums after 2005, (b) albums released on Carol before 1974, and (c)
album titled Blood Money. All other rows shall be removed.\newline

\begin{verbatim}
    DELETE FROM CD
        WHERE (Year <= 2005 AND Label != 'Carol')
            OR (Year <= 1974 AND Label = 'Carol')
            OR Title != 'Blood Money';
\end{verbatim}

\subsection*{Problem 3}

\subsubsection*{(b)}
Find all musicians that released multiple albums in 1970s. Report the name of
the musician in relational algebra. \newline

$\pi_{Name}(M \bowtie_{M.ID = C.Musician}(\sigma_{C.Year >= 1970 \land 
C.Year < 1980}(C)))$

\subsubsection*{(c)}
For each song that is longer than Shingle's duration, report its title, the year it was
released, and the name of the musician who wrote it. \newline

Shingle's duration: $\rho_{SGL}(\pi_{Duration}(\sigma_{Name = \text{Shingle}}
(S)))$

$\pi_{S.Title, C.Year, M.Name}(((S \bowtie_{S.Duration > SGL.Duration}(SGL)) 
    \bowtie_{S.CD = C.ID} C) \bowtie_{M.ID = C.Musician} M)$

\subsubsection*{(d)}
Find all albums that do not have a song longer than 4 minutes. Report the name 
of the album and the name of the musician who recorded the album. \newline

Find all songs more than 4 minutes long: $\rho_{S4}(\pi_{Duration}(
    \sigma_{Duration > 4}(S)))$

$\pi_{CD.Title, M.Name}((C \bowtie_{C.ID = S.CD} S) \bowtie_{C.Musician = M.ID}
M) - \pi_{CD.Title, M.Name}((C \bowtie_{C.ID = S.CD} S) \bowtie_{C.Musician =
M.ID} M \bowtie_{S.Duration > 4} S)$

\subsection*{Problem 4}
\subsubsection*{(a)}
Forgot to write "Report the musicians names sorted by Name" \newline

\subsubsection*{(c)}
Forgot to include Peter Hammill in the result. \newline

\subsection*{Problem 6}
\subsubsection*{(2)}
Foreign keys cannot consist of multiple attributes (T/F). \newline

\textbf{Answer:} True.

\subsubsection*{(6)}
If R and S have the same relational schema, R $\cap$ S is always equal to
$R - (R - S)$ (T/F). \newline

\textbf{Answer:} True, write out example to prove.


\end{document}
